\documentclass[a4paper]{article}

% Packages
\usepackage[14pt]{extsizes}
\usepackage[T2A]{fontenc}
\usepackage[russian]{babel}
\usepackage[left=20mm, top=15mm, right=15mm, bottom=20mm]{geometry}
\usepackage{graphicx} % For images
\usepackage{amsmath, amssymb} % For equations
\usepackage{booktabs} % For better tables
\usepackage{pgfplots} % For plotting graphs
\usepackage{xcolor} % For color support
\usepackage{caption} % For captioning tables and figures
\usepackage{float} % For precise float placement (images, tables)
\usepackage{array} % For better table management
\pgfplotsset{compat=1.17}

% Importing custom definitions (lstset, tikzset, etc.)
% % listing for programming code blocks
\lstset{
	language=C++,                 % Programming language
	basicstyle=\ttfamily\normalsize, % Adjust font size
	keywordstyle=\color{blue},    % Style for keywords
	stringstyle=\color{red},      % Style for strings
	commentstyle=\color{gray},   % Style for comments
	morecomment=[l][\color{magenta}]{\#}, % Special comment style
	breaklines=true,              % Line breaking in long lines
	numbers=left,                 % Line numbering on the left
	numberstyle=\tiny\color{gray},% Style for line numbers
	frame=single,                 % Code frame
	showstringspaces=false        % Don't show spaces in strings
}

% % tikz styles for flowcharts
\tikzset{
	startstop/.style={
			rectangle,
			rounded corners,
			minimum width=3cm,
			minimum height=1cm,
			text centered,
			draw=black,
			fill=red!30
		},
	io/.style={
			trapezium,
			trapezium left angle=70,
			trapezium right angle=110,
			minimum width=3cm,
			minimum height=1cm,
			text centered,
			draw=black,
			fill=blue!30
		},
	process/.style={
			rectangle,
			minimum width=3cm,
			minimum height=1cm,
			text centered,
			draw=black,
			fill=orange!30
		},
	decision/.style={
			diamond,
			aspect=2,
			minimum width=3cm,
			text centered,
			draw=black,
			fill=green!30
		},
	arrow/.style={
			thick,
			->,
			>=stealth
		},
	prep/.style={
			chamfered rectangle,
			chamfered rectangle xsep=2cm,
			draw,
			thick,
			minimum width=5cm,
			minimum height=1cm,
			text centered,
			text width=2.5cm,
			font=\small,
			fill=yellow!30
		},
}


% -------------------------------

\begin{document}

% Title page
\begin{center}
	\vspace{1cm}
	\large{Университет ИТМО}\\
	\large{Факультет программной инженерии и компьютерной техники}\\
	\vspace{4cm}
	\Large{\textbf{Учебно-исследовательская работа №1 (УИР 1)\\}}
	\vspace{0.3cm}
	\large{\textbf{<<Обработка результатов измерений: статический анализ числовой последовательности>>\\}}
	\vspace{-0.3cm}
	\begin{center}
		\large{по дисциплине <<Моделирование>>}
	\end{center}
	\vspace{3cm}
\end{center}
\normalsize{
	\begin{flushright}
		Выполнили:
		\par
		Студенты групп P3331 и P3334
		\par
		Дворкин Борис Александрович
		\par
		Бабенко Даниил Александрович
		\par
		\textbf{Вариант: 35}
		\par
		\vspace{1cm}
		Преподаватель:
		\par
		Авксентьева Елена Юрьевна
	\end{flushright}
}\\
\vspace{6cm}
\begin{center} г. Санкт-Петербург
	\par
	2024 г.
\end{center}
\thispagestyle{empty}
\thispagestyle{empty}

\thispagestyle{empty}

\newpage
\pagestyle{plain}
\setcounter{page}{1}

% -------------------------------

% autogenerated table of contents
\linespread{0.9}
\tableofcontents
\linespread{1}

% -------------------------------

\newpage
\section*{Цель работы}
Изучение методов обработки и статистического анализа результатов измерений на примере
заданной числовой последовательности путем оценки числовых моментов и выявления свойств
последовательности на основе корреляционного анализа, а также аппроксимация закона
распределения заданной последовательности по двум числовым моментам случайной величины.

% -------------------------------

\section{Рассчитать числовые моменты заданной ЧП}
\begin{itemize}
	\item Математическое ожидание:
	      \[
		      \mu = \frac{1}{n} \sum_{i=1}^{n} X_i
	      \]
	      где \( X_i \) — элементы последовательности, \( n \) — количество элементов.

	\item Дисперсия:
	      \[
		      \sigma^2 = \frac{1}{n-1} \sum_{i=1}^{n} (X_i - \mu)^2
	      \]
	      где \( \mu \) — математическое ожидание, \( n-1 \) — степень свободы для корректировки выборки.

	\item Среднеквадратическое отклонение:
	      \[
		      \sigma = \sqrt{\sigma^2}
	      \]

	\item Коэффициент вариации:
	      \[
		      CV = \frac{\sigma}{\mu} \times 100
	      \]
	      Коэффициент вариации показывает относительное отклонение данных от среднего значения в процентах.
\end{itemize}


\section{Построение доверительных интервалов}

Пусть для некоторой случайной величины \( X \) в результате имитационного моделирования получена несмещенная оценка математического ожидания \( \tilde{m} \) и оценка дисперсии \( \tilde{D} \), рассчитанные по следующим формулам:

\[
\tilde{m} = \frac{1}{n} \sum_{i=1}^{n} X_i
\]
\[
\tilde{D} = \frac{1}{n-1} \sum_{i=1}^{n} (X_i - \tilde{m})^2
\]

где \( X_i \) — значения случайной величины, \( n \) — количество измерений. 

Для построения доверительного интервала с доверительной вероятностью \( p \) определим величину погрешности \( \epsilon_p \), которая вычисляется по формуле:

\[
\epsilon_p = t_p \cdot \sigma_m = t_p \cdot \frac{\sqrt{\tilde{D}}}{\sqrt{n}}
\]

где \( t_p \) — коэффициент, найденный по таблице для выбранного уровня доверия \( p \), \( \sigma_m \) — оценка стандартного отклонения математического ожидания.

\subsection{Доверительный интервал для математического ожидания}

Доверительный интервал для математического ожидания с заданной доверительной вероятностью \( p \) задается как:

\[
[m_\text{н}, m_\text{в}] = [\tilde{m} - \epsilon_p, \tilde{m} + \epsilon_p]
\]

где:
\begin{itemize}
    \item \( m_\text{н} = \tilde{м} - \epsilon_p \) — нижняя граница доверительного интервала,
    \item \( m_\text{в} = \tilde{м} + \epsilon_p \) — верхняя граница доверительного интервала.
\end{itemize}

Таким образом, с вероятностью \( p \) истинное значение математического ожидания находится в интервале \( [m_\text{н}, m_\text{в}] \).

\subsection{Относительная погрешность}

Относительная погрешность оценки математического ожидания \( \delta \) рассчитывается по следующей формуле:

\[
\delta = \frac{\epsilon_p}{\tilde{m}} \times 100
\]

где \( \epsilon_p \) — половина длины доверительного интервала. Это значение показывает, насколько сильно точечная оценка отличается от истинного значения в процентах.

\subsection{Пример расчета}

Предположим, что в результате 300 измерений случайной величины \( X \) были получены следующие оценки:
\[
\tilde{m} = 50, \quad \tilde{D} = 45816
\]
Тогда оценка стандартного отклонения математического ожидания:

\[
\sigma_m = \frac{\sqrt{\tilde{D}}}{\sqrt{300}} = 28.57
\]

Для доверительной вероятности \( p = 0.95 \), коэффициент \( t_p = 1.960 \), тогда величина погрешности:

\[
\epsilon_p = 1.960 \times 28.57 = 56.99
\]

Доверительный интервал:

\[
m_\text{н} = 50 - 56.99 = -6.99, \quad m_\text{в} = 50 + 56.99 = 106.99
\]

Относительная погрешность:

\[
\delta = \frac{56.99}{50} \times 100 = 113.98\%
\]

Таким образом, доверительный интервал для математического ожидания составляет \( [-6.99, 106.99] \), а относительная погрешность — \( 113.98\% \).

% -------------------------------


\section{Введение и статистический анализ числовой последовательности}
\begin{enumerate}
	\item Рассчитать следующие числовые моменты для заданной последовательности:
	      \begin{itemize}
		      \item Математическое ожидание.
		      \item Дисперсия.
		      \item Среднеквадратическое отклонение.
		      \item Коэффициент вариации.
	      \end{itemize}
	\item Построить доверительные интервалы для математического ожидания при доверительных вероятностях 0.9, 0.95 и 0.99.
	\item Рассчитать относительные отклонения в процентах от эталонных значений (выборка из 300 элементов).
\end{enumerate}

\section{Графический анализ последовательности}
\begin{enumerate}
	\item Построить график значений числовой последовательности.
	\item Определить характер последовательности (возрастающая, убывающая, периодическая).
	\item Выполнить автокорреляционный анализ и оценить, можно ли считать последовательность случайной.
\end{enumerate}

\section{Анализ распределения}
\begin{enumerate}
	\item Построить гистограмму распределения частот для последовательности.
	\item Выполнить аппроксимацию распределения последовательности по двум моментам, выбрать одно из следующих распределений:
	      \begin{itemize}
		      \item Равномерное.
		      \item Экспоненциальное.
		      \item Нормированное Эрланга.
		      \item Гиперэкспоненциальное.
	      \end{itemize}
\end{enumerate}

\section{Генерация случайной последовательности}
\begin{enumerate}
	\item Реализовать генератор случайных величин в соответствии с выбранным законом распределения.
	\item Сгенерировать новую последовательность случайных величин.
	\item Рассчитать числовые моменты для сгенерированной последовательности.
\end{enumerate}

\section{Сравнительный анализ}
\begin{enumerate}
	\item Построить графики и гистограммы для сравнения исходной и сгенерированной последовательностей.
	\item Провести автокорреляционный анализ сгенерированной последовательности.
	\item Оценить корреляционную зависимость исходной и сгенерированной последовательностей.
\end{enumerate}

\section{Оформление отчёта}
\begin{enumerate}
	\item Оформить результаты расчётов в виде таблиц (характеристики для 10, 20, 50, 100, 200 и 300 величин).
	\item Включить графики и гистограммы в отчёт.
	\item Сделать выводы по результатам анализа и аппроксимации.
\end{enumerate}

% -------------------------------


% -------------------------------

\section{Выводы по работе}

Вывод по работе

\end{document}
