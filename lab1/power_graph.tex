\section{Графики зависимости $P_n(I_n)$ и $\eta(I_n)$}
\subsection{График зависимости мощности в нагрузке \( P_n(I_n) \)}

На рисунке 3 представлена зависимость мощности в нагрузке \( P_n \) от тока \( I_n \).
Мощность в нагрузке \( P_n \) растёт с увеличением тока \( I_n \) до определённого предела,
после чего начинает снижаться, что объясняется возрастанием сопротивления нагрузки
и уменьшением напряжения на ней.

\begin{figure}[h]
	\centering
	\begin{tikzpicture}
		\begin{axis}[
			width=17cm, height=13cm, % Размер графика
			xlabel={$I_n$~[мА]}, % Ось X
			ylabel={$P_n$~[Вт]},  % Ось Y
			axis lines=middle,
			grid=major,          % Основная сетка
			xmin=0, xmax=22,     % Диапазон по оси X
			ymin=0, ymax=0.06,   % Диапазон по оси Y
			thick,               % Толщина линии
			% scaled ticks=false,
			legend style={at={(0.95,0.95)}, anchor=north east}, % Легенда
			label style={font=\small},
			tick label style={font=\small},
			xtick={0, 2.5, 5, 7.5, 10, 12.5, 15, 17.5, 20}, % Подписи по оси X
			ytick={0, 0.01, 0.02, 0.03, 0.04, 0.05, 0.06}, % Подписи по оси Y
			]

			% Зависимость мощности P_n от I_n
			\addplot[color=blue, thick, mark=*] coordinates {
					(0, 0.00)
					(1.98, 0.021)
					(3.96, 0.038)
					(5.94, 0.049)
					(7.92, 0.056)
					(9.9, 0.059)
					(11.88, 0.056)
					(13.864, 0.049)
					(15.84, 0.038)
					(17.806, 0.021)
					(20, 0.000)
				};
			\addlegendentry{Мощность \( P_n(I_n) \)}

		\end{axis}
	\end{tikzpicture}
	\caption{График зависимости мощности в нагрузке \( P_n(I_n) \)}
\end{figure}
